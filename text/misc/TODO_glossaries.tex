
\section{Terminology}
\label{sec:pdm_terminology}

\paragraph{Subject}

Subject is what is being monitored and what is, ultimately, to be maintained.
It may be a whole system such as a power plant, a machinery such as a compressor or a specific component such as a bearing.
Other frequently used terms: asset, unit under test (UUT), machine, equipment

\paragraph{Health}
Health is an ability of the subject to operate, i.e. its ability to fulfill its purpose.

\paragraph{Fault}
Fault is a change in the subject that negatively effects its health.
When a subject suffers from a fault, it is said to be malfunctioning.
A fault typically starts at a certain time point and grows in severity over time.
There might be several different fault that can occur to the subject --- fault types.
For example, typical faults types of bearings are inner race fault, outer race fault and rolling  element (ball) fault.

\paragraph{Health Label}

Health label is a representation of the subject's health as a categorical variable.
The health label can be either binary, e.g. healthy or faulty, or multiclass, e.g. healthy or one of fault types.
Synonyms: condition, condition class, condition label.

\paragraph{Failure}
Failure is a special health state of the subject when it is unable to operate.
It is a typically a reason for performing a corrective maintenance, i.e. a repair.
 - end of life (EoL)

\paragraph{Health Indicator}
Health indicator is a measure of health as a continuous variable which represent.
loss of health, e.g. loss of battery capacity or amount of milling tool wear.
A health indicator can be also perceived as a measure of fault's severity.
 - Health index

\paragraph{Failure Threshold}
Failure threshold is a value of health indicator that is no longer considered as safe
or permissible for operation.

\paragraph{Remaining Useful Life}
Remaining useful life of a subject is a special case of a health indicator which represents an absolute value of time remaining to failure.
 - time to failure

\paragraph{Fault Detection}
 - Condition classification
 
\paragraph{Fault detection, isolation and recovery}